\documentclass[a4paper,12pt]{report}

\usepackage{authblk}
\usepackage{hyperref}
%French-specific commands
%--------------------------------------
\usepackage[french]{babel}
\usepackage[autolanguage]{numprint} % for the \nombre command
\usepackage{titlesec}
\hypersetup{
    colorlinks,
    citecolor=black,
    filecolor=black,
    linkcolor=black,
    urlcolor=black
}


\titleformat{\chapter}[display]
  {\normalfont\huge\bfseries} % Style de titre
  {} % Préfixe vide
  {0pt} % Espacement entre le titre et le contenu
  {\huge} % Style du titre

\renewcommand{\chaptername}{}


\title{Rapport Projet réseaux}
\author{
    Lucas BÉRANGER \and
    Gillian LE PÉVÉDIC \and
    François BESNARD \and
    Alexandre FLOURY \and
}
\date{30/10/2024}

\begin{document}

    \maketitle  
    \newpage

    \tableofcontents
    \newpage

    \chapter{Introduction}
        \section{Présentation de l'équipe}
        \section{Répartition des tâches}


    \chapter{Architecture du réseau}
        \section{Vlan}
        \section{DHCP}
            Dans les consignes, il était demandé de mettre en place un service DHCP qui permettait de fournir une adresse IP à toutes les machines du réseau. 
            
            Pour cela, le service isc-dhcp-server est déjà présent sur les distributions Ubuntu.
            \subsection{Configuration serveur}
            Pour configurer le serveur DHCP, il faut modifier le fichier /etc/dhcp/dhcpd.conf. 
            Dans ce fichier, chaque sous-réseaux/vlans doit être déclaré comme suit :
                
            \begin{verbatim}
            #Production
            subnet 192.168.10.0 netmask 255.255.255.0 {
                range 192.168.10.1 192.168.10.250;
                option routers 192.168.10.254;
                option broadcast-address 192.168.10.255;
            }
            \end{verbatim}
                
            Ici, le sous-réseau Production a une plage d'adresse allant de 192.168.10.1 jusqu'à 192.168.10.250 et une passerelle par défaut à 192.168.10.254.

            Il a été choisi de partir sur des plages d'adresses de 250 adresses pour chaque sous-réseau, ce qui fait qu'il y a un maximum de 250 machines par sous-réseau, 3 adresses sont gardées en réserve.
            Tous les sous-réseaux Commercial, Administration, Production et la DMZ sont des classes C et ont été configurés de la même manière avec leur plage d'adresses correspondant à leur vlan respectif.

            Certaines machines doivent reçevoir la même adresse IP à chaque fois qu'elles se connectent au réseau (par exemple le serveur web).
            Pour cela, il faut déclarer des machines en utilisant leur adresse MAC dans le fichier de configuration du serveur DHCP comme suit:

            \begin{verbatim}
            #Serveur Web
            host ServeurWeb {
                hardware ethernet 08:00:a0:24:21:02;
                fixed-address 192.168.100.250;
            }
            \end{verbatim}

            Avec cette partie du fichier, la machine Serveur Web reçevra toujours l'adresse IP 192.168.100.250.

            Une fois le fichier de configuration édité, il faut redémarrer le service DHCP pour que les modifications soient prises en compte:
            \begin{verbatim}
            'sudo systemctl restart isc-dhcp-server'
            \end{verbatim}
            et vérifier le status du service pour s'assurer qu'il fonctionne correctement:
            \begin{verbatim}
            'sudo systemctl status isc-dhcp-server'
            \end{verbatim}

            \subsection{Configuration DHCP client}
            Pour configurer un client et lui faire récupérer une adresse IP via le serveur DHCP, il suffit de modifier le fichier /etc/network/interfaces en ajoutant les lignes suivantes:
            \begin{verbatim}
            #Exemple avec une interface enp0s3
            auto enp0s3
            iface enp0s3 inet dhcp
            \end{verbatim}

            Si plutard il y a besoin d'ajouter une nouvelle interface, il suffit de rajouter les lignes ci-dessus en changeant le nom de l'interface.
            Si le serveur est correctement configuré ainsi que le réseau, il reste à exécuter '\textbf{sudo dhclient}' afin de forcer le client à demander une adresse IP au serveur DHCP.
            Une fois l'exécution de la commande terminée, la commande '\textbf{ip a}' permet de vérifier que le client a bien reçu une adresse IP.

        \section{Firewall}
        \section{}


    \chapter{Conclusion}
        \section{Difficultés rencontrées}
        \section{Pistes d'amélioration}

    \chapter*{Annexes}
        \section{Diagramme de Flux}
        \section{Schéma réseau logique}
        \section{Schéma réseau physique}
        \section{Plan d'adressage}
        




\end{document} 