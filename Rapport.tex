\documentclass[a4paper,12pt]{report}

\usepackage{authblk}
\usepackage{hyperref}
%French-specific commands
%--------------------------------------
\usepackage[french]{babel}
\usepackage[autolanguage]{numprint} % for the \nombre command
\usepackage{titlesec}
\hypersetup{
    colorlinks,
    citecolor=black,
    filecolor=black,
    linkcolor=black,
    urlcolor=black
}


\titleformat{\chapter}[display]
  {\normalfont\huge\bfseries} % Style de titre
  {} % Préfixe vide
  {0pt} % Espacement entre le titre et le contenu
  {\huge} % Style du titre

\renewcommand{\chaptername}{}
 

\title{Rapport Projet réseaux}
\author{
    Lucas BÉRANGER \and 
    Gillian LE PÉVÉDIC \and
    François BESNARD \and
    Alexandre FLOURY \and
}
\date{30/10/2024}

\begin{document}

    \maketitle  
    \newpage

    \tableofcontents
    \newpage

    \chapter{Introduction}
        \section{Présentation de l'équipe}
        \section{Répartition des tâches}


    \chapter{Architecture du réseau}
        \section{Vlan}
        \section{DHCP}
        \section{Firewall}
        \section{Switch}
        Nous avons configuré des switchs Cisco 2960 Series pour assurer la redondance entre deux bâtiments sur cinq VLANs : 10, 20, 50, 60, et 100. Chaque switch a été configuré avec des ports en mode access et trunk pour permettre une communication efficace et sécurisée entre les différents segments du réseau. Les configurations incluent la définition des VLANs, l'attribution des ports aux VLANs, et la configuration des trunks pour permettre le passage des VLANs nécessaires.

    \chapter{Conclusion}
        \section{Difficultés rencontrées}
        \section{Pistes d'amélioration}

    \chapter*{Annexes}
        \section{Diagramme de Flux}
        \section{Schéma réseau logique}
        \section{Schéma réseau physique}
        \section{Plan d'adressage}
        




\end{document} 